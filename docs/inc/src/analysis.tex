\section{Аналитическая часть}

В данном разделе проводится описание объектов сцены, анализ существующих алгоритмов для решения поставленной задачи. В результате анализа выбирается алгоритм для дальнейшей реализации.

\subsection{Формализация объектов синтезируемой сцены}

Сцена состоит из следующего набора объектов:

\begin{enumerate}
	\item точечный источник света --- задается положением в пространстве, интенсивностью и цветом. В зависимости от характеристик источника определяется освещенность объектов сцены. Характеристики имеют значения по умолчанию, но имеется возможность их изменить;
	\item камера --- задается положением в пространстве, углом поворота вокруг каждой из трех координатных осей;
	\item линза --- задается диаметром линзы и радиусом кривизны обеих поверхностей;
	\item объекты сцены --- примитивы из заданного набора: 
	\begin{itemize}
		\item параллелепипед --- задается длиной сторон по каждой координатной оси;
		\item призма трехгранная --- задается длиной стороны основания и высотой;
		\item сфера --- задается радиусом;
		\item пирамида четырехгранная --- задается длиной стороны основания и высотой.
	\end{itemize}
	Помимо специфичных для каждого примитива свойств, всем возможно задать положение в пространстве.
\end{enumerate}

\subsection{Физика линзы}

\textit{Линзой} называется прозрачное тело, ограниченное двумя сферическими поверхностями, линия центров этих поверхностей называется главной оптической осью линзы \cite{msu_optic}.

\img{60mm}{lens}{Ход луча через линзу \cite{msu_optic}}

Рассмотрим точечный источник света $S$, расположенный вне линзы с показателем преломления $n$ и находящийся на ее главной оптической оси (рисунок \ref{img:lens}). Путь произвольного луча, исходящего от источника в направлении линзы, будет проходить через точки $М$ и $N$ с двумя последовательными преломлениями, этот луч пересечет главную оптическую ось в точке $S'$, таким образом $S'$ будет действительным изображением источника. Используя законы оптики и полагая, что расстояние между поверхностями $l \rightarrow 0$ (пренебрежимо мало), можно получить так называемую \textit{формулу тонкой линзы}:

\begin{equation}
	\label{eq:thin_lens}
	\frac{1}{d} + \frac{1}{f} = (n-1)\left(\frac{1}{R_1} - \frac{1}{R_2}\right)
\end{equation}

Формула тонкой линзы весьма применима в реальных расчетах, так как достаточно точно позволяет предсказать ход лучей при маленькой толщине линзы. Однако она не подходит для решения поставленной задачи, так как в общем случае линза не обязательно тонкая. Таким образом требуется просчитывать изображение с учетом преломления света через обе поверхности линзы.


\subsection{Описание методов}
\subsubsection{Глобальное освещение}

Разные материалы, используемые для создания объектов на сцене, взаимодействуют с освещением по-разному. В процессе этого взаимодействия часть энергии отражается, часть преломляется, а оставшаяся часть поглощается. Существует также сценарий, когда сам материал излучает свет. Когда луч света исходит от источника и попадает на поверхность объекта, возможно его отражение. Этот отраженный луч может затем попасть на поверхность другого объекта, претерпев переотражения. \textit{Первичный луч} --- тот, который напрямую исходит от источника света, в то время как  \textit{вторичный луч} --- тот, который претерпел одно или несколько отражений. Построение физической модели света, учитывающей только первичное освещение (первичные лучи), может существенно снизить качество синтезируемых изображений.

Ещё одним примером упрощения физической модели является отказ от учета теней в сцене. Модели, которые игнорируют этот процесс, называются \textit{локальными}. В противоположность им, модели, учитывающие передачу света между поверхностями, обозначаются как \textit{глобальные} или \textit{модели глобального освещения} \cite{levedeb_gi_history}.

\subsubsection*{Диффузное отражение}

Пусть луч падает в точку $P$ в направлении $i$ и отражается в направлении $r$, преломляется в направлении $t$, а вектором нормали к поверхности является $n$ (рисунок \ref{img:ray_reflect_refract}). Все вектора единичные.

\img{80mm}{ray_reflect_refract}{Преломление и отражение луча \cite{boreskov_cg}}

Идеальное диффузное отражение описывается законом Ламберта, который утверждает, что свет, падающий на поверхность, равномерно рассеивается во всех направлениях. Таким образом, нет определенного направления, в котором отражается луч свет, все направления считаются равными, и освещенность точки пропорциональна доле площади, видимой от источника света, то есть скалярному произведению вектора $( \vec{i}, \vec{n} )$ \cite{boreskov_cg}. Можно задать уравнение для вычисления количества света, диффузно отраженного точкой $P$ как:

\begin{equation}
	\label{eq:diffuse_light}
	I_d = \sum_{k=1}^{N} I_k \left(\vec{n}, \vec{i}_k\right),
\end{equation}

где $N$ --- количество попадающих в точку лучей, $I_k$ --- интесивность $k$-го луча.

\subsubsection*{Зеркальное отражение}

Зеркальное отражание характеризует меру \textit{отполированности} поверхности. Пусть $\vec{l}$ --- \textit{вектор обзора}, единичный, направленный из точки $P$ к наблюдателю. Учитывая, что вектора $\vec{i}$, $\vec{i}$ и $\vec{r}$ лежат в одной плоскости, легко можно выразить направление отраженнего луча:

\begin{equation}
	\label{eq:reflected_ray}
	\vec{r} = \vec{i} - 2(\vec{i},\vec{n})\vec{n},
\end{equation}

тогда уравнение для вычисления зеркально отраженного количества света точкой $P$ можно задать как:

\begin{equation}
	\label{eq:specular_light}
	I_s = \sum_{k=1}^{N} {\left( I_k \left(\vec{r}_k, \vec{l}\right) \right)}^s,
\end{equation}

где $N$ --- количество попадающих в точку лучей, $I_k$ --- интесивность $k$-го луча, $\vec{r}_k$ - направление отраженного $k$-го луча, $s$ --- \textit{показатель отражения} \cite{gambetta_cg}.

\subsubsection*{Преломление}

Согласно закону Снеллиуса векторы $\vec{i}$, $\vec{n}$ и преломленный вектор $\vec{t}$ лежат в одной плоскости и для углов справедливо соотношение:

\begin{equation}
	\label{eq:snell_law}
	\eta_i\sin{\theta_i} = \eta_t\sin{\theta_t},
\end{equation}

где $\eta_i$ и $\eta_t$ --- коэффициенты преломления сред из которой пришел луч и в которую преломился соответственно. Исходя из этого можно получить формулу для нахождения $\vec{t}$:


\begin{equation}
	\label{eq:snell_law_vector_form}
	\vec{t} = \frac{\eta_i}{\eta_t}  \vec{i} + \left( \frac{\eta_i}{\eta_t} C_i - \sqrt{1 + \left(\frac{\eta_i}{\eta_t}\right)^2 ( C_i^2 - 1)} \right) \vec{n},
\end{equation}

где $C_i = \cos{\theta_i} = -( \vec{i}, \vec{n} )$. В таком случае ситуация, при которой выражение под корнем $1 + \left(\frac{\eta_i}{\eta_t}\right)^2 ( C_i^2 - 1) < 0$ соответствует явлению полного внутреннего отражения, когда световая энергия отражается от границы раздела сред и преломления фактически не происходит.

\subsubsection{Алгоритм обратной трассировки лучей}

Для синтеза изображения требуется рассчитать цвет каждого пикселя с учетом общей световой энергии, достигающей соответствующего светочувствительного рецептора наблюдателя \cite{nikulin_optic_effects}. Наблюдатель воспринимает объект, когда к нему доходят лучи, путь которых начинается на источнике света и заканчивается на наблюдателе. В соответствии с этим можно использовать свойство обратимости световых лучей и испускать их в обратном направлении --- от наблюдателя к объектам сцены. Путь луча при этом будет строиться согласно тем же законам оптики, что и при \textit{прямом} направлении.

Наблюдатель (также называемый \textit{камерой} \cite{gambetta_cg}) находится в некоторой точке $S$, смотрит через \textit{окно просмотра} (виртуальная проективная плоскость \cite{nikulin_optic_effects}). Каждый пиксель на итоговом изображении соответствует некоторой точке $P$ окна просмотра. Таким образом, представляя точки в виде радиус-векторов, направление испускаемого луча можно найти как: 

\begin{equation}
	\label{eq:ray_tracing_direction}
	\vec{V} = \vec{P} - \vec{S}.
\end{equation}

Выпущенный луч (называемый \textit{первичным}) в результате может пересечься или не персечься с каким-либо объектом (рисунок \ref{img:camera_window}). В случае не попадания ни в один объект сцены, пиксель на изображении, соответствующий выпущенному лучу, принимает цвет фона.

\img{150mm}{camera_window}{Алгоритм обратной трассировки лучей \cite{gambetta_cg}}

При пересечении лучем объекта могут быть порождены \textit{вторичные лучи} в результате преломления или отражения, а цвет соответствующего пикселя может быть рассчитан по следующей формуле \cite{boreskov_cg, rogers, whitted_illum}:

\begin{equation}
	\label{eq:ray_tracing_color}
	C = k_a I_a C_p + k_d C_p \sum_{j} I_j (\vec{n}, \vec{i}_j) + k_s \sum_{j} I_j (\vec{l}, \vec{r}_j)^s + k_t I_t,
\end{equation}

где $I_a$ --- интенсивность \textit{фонового} освещения, $I_j$ --- интенсивность $j$-го источника света, $I_t$ --- освещенность, переносимая преломленным лучом, $C_p$ --- цвет в точке пересечения луча и объекта, $k_a$ --- коэффициент фонового освещения, $k_d$ --- коэффициент диффузного освещения, $k_s$ --- коэффициент зеркального освещения, $k_t$ --- вклад преломленного луча, $\vec{n}$ --- вектор внешней нормали к поверхности в точке пересечения луча и объекта, $\vec{i}_j$ --- единичный вектор направления из точки пересечения на $j$-й источник света, $\vec{r}_j$ --- единичный вектор направления отраженного луча, $\vec{l}$ --- единичный \textit{вектор обзора}, направленный из точки пересечения к наблюдателю, $s$ --- показатель отражения.


\subsubsection{Алгоритм трассировки фотонов}

Алгоритм обратной трассировки лучей достигает значительных результатов, позволяя моделировать отражение и преломление. Несмотря на широкую популярность и эффективность этого метода, существует набор физических явлений, которые он описывает недостаточно точно или вовсе не охватывает. Примерами таких явлений являются рассеивающие отражения, проявляющиеся в изменении цвета при отражении света от цветной поверхности (цветовой оттенок от комода из красного дерева на белом ковре), а также эффект сфокусированного света, известного как каустики, видимые, например, в бликах от воды на дне бассейна \cite{mezhenin_3d}.

Алгоритм трассировки фотонов \cite{kalos_phd} (также известен как алгоритм прямой трассировки лучей \cite{jensen_gi}) представляет собой процесс в котором лучи (фотоны) исходят из источника света, который может быть не точечным.  Пример прохождения лучей в алгоритме приведен на рисунке \ref{img:photon_tracing}. В данном алгоритме лучи при столкновении с поверхностью могут породить вторичные лучи, если поверхность имеет преломление или \textit{идеальное} (зеркальное) отражение. Если же поверхность не имеет способности породить вторичные лучи, считается, что энергия луча поглощается и его дальнейшая трассировка прекращается. Серьезным недостатком алгоритма является низкий коэффициент полезного действия --- большинство испускаемых фотонов не попадают в глаз наблюдателя.

\img{90mm}{photon_tracing}{Путь лучей в алгоритме трассировки фотонов \cite{kalos_phd}}

\subsubsection{Алгоритм фотонных карт}

Концептуально алгоритм фотонных карт \cite{jensen_gi} немного отличается от остальных методов трассировки, так как является гибридным. Помимо непосредственной трассировки света от источника, он включает этапы построения фотонной карты и сбора освещенности. Фотоны при таком подходе считаются переносчиками некоторой дискретной порции световой энерии. На первом этапе из источника света испускаются фотоны, в зависимости от свойств материала, могут произойти разные события: фотон может отразиться диффузно (в случайном направлении), зеркально, преломиться через поверхность или поглотиться (рисунок \ref{img:photon_mapping}). Только при диффузном отражении запись о фотоне заносится в список.

\img{100mm}{photon_mapping}{Пути фотонов на сцене: (а) два диффузных отражения и поглощение, (б) зеркальное отражение и два диффузных отражения, (в) два преломления и поглощение \cite{jensen_gi}}

Будет ли фотон отражен, преломлен, или поглощен определяется по методу \textit{русской рулетки} \cite{jensen_gi}. Для монохроматического света, коэффициента диффузного отражения $d$ и коэффициента зеркального отражения $s$ (при $d+s\leq1$) используется равномерно распределенная в промежутке $[0,1]$ случайная величина $\xi$:

\begin{equation}
	\begin{aligned}
		&&\xi \in [0,d] \quad\ \  && \rightarrow && \text{диффузное отражение} \\
		&&\xi \in [d,s + d] && \rightarrow && \text{зеркальное отражение} \\
		&&\xi \in [s+d,1] && \rightarrow && \text{поглощение}
	\end{aligned}
\end{equation}

Сумма энергий фотонов на поверхностях формирует фотонную карту --- некоторую структуру пространственного разбиения для удобства нахождения фотонов на следующем этапе. На стадии сбора освещенности происходит сбор энергии из фотонных карт, который может быть выполнен различными подходами. Обычно осуществляется сбор по сфере для каждой поверхности объекта. В каждой точке сбора рассматривается окрестность, и значения точек фотонной карты, попавших в эту окрестность, суммируются и нормируются, представляя результирующую освещенность в точке. Точки сбора сами по себе определяются другими методами, например обратной трассировкой лучей. Помимо этого, другие оптические эффекты как отражения, преломления, тени, также не рассчитываются алгоритмом фотонных карт, он нужен только для вычисления непрямой освещенности.

Алгоритм фотонных карт характеризуется высокой вычислительной сложностью, которая компенсируется качеством сгенерированных изображений. Обычно он применяется для расчета каустик. Важным аспектом в самом алгоритме является выбор радиусов сфер для сбора, где более большой радиус приводит к длительному сбору и размытости изображения, в то время как маленький радиус дает больше шума. Также требуется построение ускоряющих структур (например $KD$-дерева) для более эффективного поиска точек в фотонной карте.

\subsubsection{Алгоритм трассировки путей}

В алгоритме трассировки путей \cite{kalos_phd} при встрече луча с поверхностью происходит излучение двух новых лучей: один направлен в произвольном (метод русской рулетки) направлении, а другой направлен к источнику света. Прохождение лучей в алгоритме трассировки путей показано на рисунке \ref{img:path_tracing}. Доля энергии, отраженной при этом, рассчитывается на основе двулучевой функции отображения (ДФО) --- характеристики конкретного материала, показывающей долю энергии, отраженную в зависимости от длины волны, направления на источник света и камеру \cite{ilyin_modeling, kalos_phd}:
\begin{equation}
	\label{eq:DFO}
	\text{ДФО} = \frac{L_0}{L_i \cos(\phi) dw},
\end{equation}

где $L_0$ --- количество энергии, отраженное в направлении к наблюдателю. $L_i$ --- количество энергии, приходящее от источника. $\phi$ --- угол между нормалью к поверхности и вектором на источник. $dw$ --- дифференциальный телесный угол, порожденный направлением к источнику. Работа алгоритма завершается, если длина пройденного пути достигает заданной константы, называемой глубиной трассировки. 

\img{90mm}{path_tracing}{Путь лучей в алгоритме трассировки путей \cite{kalos_phd}}



\clearpage

\subsection{Сравнение алгоритмов}

В таблице \ref{tbl:comp} приведено сравнение алгоритмов по возможности симулировать оптические эффекты и некоторым другим характеристикам. <<+>> означает, что алгоритм имеет возможность строить изображение с учетом оптического эффект, <<->> --- не имеет возможности. Обозначения:

\begin{itemize}
	\item ОТЛ --- алгоритм обратной трассировки лучей;
	\item ТФ --- алгоритм трассировки фотонов;
	\item ФК --- алгоритм фотонных карт;
	\item ТП --- алгоритм трассировки путей;
	\item ИС --- источник света;
	\item К --- камера;
	\item $N$ --- число вторичных лучей, которые может породить первичный луч при столкновении с поверхностью.
\end{itemize}

\begin{table}[htbp]
	\centering
	\caption{Сравнение алгоритмов}
	\label{tbl:comp}
	\begin{tabular}{|l|c|c|c|c|}
	\hline
	& ОТЛ & ТФ & ФК & ТП \\
	\hline
	Диффузное отражение & + & + & + & + \\
	\hline
	Зеркальное отражение & + & + & + & + \\
	\hline
	Преломление & + & + & + & + \\
	\hline
	Тени & + & + & + & + \\
	\hline
	Каустики & - & + & + & - \\
	\hline
	Рассеянные источники & - & + & + & + \\
	\hline
	Непрямое освещение & - & + & + & + \\
	\hline
	$N$  & $[0,2]$ & $[0,2]$ & $[0,1]$ & $[0,2]$ \\
	\hline
	Источник луча & К & ИС & ИС & К \\
	\hline
	\end{tabular}
\end{table}

Алгоритм обратной трассировки лучей является наименее вычислительно сложным, он достаточно подходит для реализации требуемой сцены, так как учитывает диффузное отражение, зеркальное отражение и преломление света.

\subsection*{Вывод}
В данном разделе было проведено описание объектов сцены, рассмотрены алгоритмы визуализации сцены, отвечающие заданным требованиям. В качестве алгоритма для реализации был выбран алгоритм обратной трассировки лучей как решающий задачу визуализации сцены с учетом диффузного и зеркального отражения и преломления с наименьшими затратами вычислительных ресурсов.




\clearpage
