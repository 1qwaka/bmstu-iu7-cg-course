\section{Конструкторская часть}

\subsection{Требования к программному обеспечению}

Программа должна предоставлять следующие возможности:

\begin{itemize}
	\item добавление примитива из набора и задание ему характеристик;
	\item изменение положения камеры в пространстве;
	\item изменение положения линзы в пространстве;
	\item изменение характеристик источника света.
\end{itemize}


\subsection{Описание типов и структур данных}

В работе будут использованы следующие типы и структуры данных:

\begin{itemize}
	\item точка --- вектор из трех вещественных чисел соответствующих координатам по каждой из осей трехмерной декартовой системы координат;
	\item цвет --- вектор из трех целых чисел, принимающих значение в диапазоне $[0,255]$, соответствует цветовой модели RGB;
	\item материал --- представляет характеристики поверхности объекта сцены, содержит цвет, коэффициенты зеркального и диффузного отражения, коэффициент преломления;
	\item сфера --- объект сцены, содержит точку центра и радиус, материал;
	\item треугольник --- основная фигура для задания объектов сцены, является набором из трех точек;
	\item многогранник --- примитивный многогранник, является набором из треугольников, также содержит материал поверхности;
	\item источник света --- содержит цвет, интенсивность и точку своего положения в пространстве;
	\item камера --- содержит точку своего положения в пространстве и углы поворота вокруг каждой из координатных осей, высоту и ширину окна просмотра, а также удаленность окна просмотра;
	\item линза --- содержит точку своего центра, показатель преломления, углы поворота вокруг каждой из координатных осей;
	\item сцена --- хранит в себе набор всех многогранников и сфер, камеру, источник света, линзу.
\end{itemize}


\subsection{Алгоритм обратной трассировки лучей}

Алгоритм обратной трассировки лучей испускает лучи от наблюдателя к объектом сцены через окно просмотра. Каждый луч проходит через окно просмотра и рассчитывает цвет некоторого пикселя на результирующем изображении. Если задать высоту и ширину изображения как $C_w$ и $C_h$, высоту и ширину окна просмотра как $V_w$ и $V_h$, удаленность окна просмотра вдоль оси $z$ как $d$, камеру в точке $O$, то точку $P_{ij}$ в плоскости окна просмотра, соответствующую пикселю изображения в $i$ строке и $j$ столбце можно найти по следующей формуле:

\begin{equation}
	P_{ij} = \left(i \frac{V_w}{C_w} + O_x, j \frac{V_h}{C_h} + O_y, d + O_z\right).
\end{equation}

Данная формула по сути является преобразованием масштаба и сдвига. Тогда вектор направления луча $\vec{D}$ можно определить по формуле:

\begin{equation}
	\vec{D} = \vec{P_{ij}} - \vec{O}.
\end{equation}

Имея точку начала луча и вектор направления, луч записывается в параметрическом виде:

\begin{equation}
	\label{eq:ray_formula}
	\vec{L} = \vec{O} + t\vec{D},
\end{equation}

где $t \in [0, +\inf]$.\\

Схема алгоритма обратной трассировки лучей приведена на рисунках \ref{img:rt1} и \ref{img:rt2}.

\img{200mm}{rt1}{Схема алгоритма обратной трассировки лучей}

\img{200mm}{rt2}{Схема алгоритма обратной трассировки лучей (продолжение)}

\clearpage

\subsubsection{Пересечение сферы и луча}

Сфера - множество точек, лежащих на некотором заданном расстоянии (\textit{радиусе}) от точки центра сферы, соответственно считая точку $C$ --- центром, $r$ --- радиусом, а $P$ --- точкой на сфере, по свойствам скалярного произведения можно задать уравнение сферы как:


\begin{equation}
	\label{eq:sphere_formula}
	( \vec{P} - \vec{C} , \vec{P} - \vec{C}) = r^2.
\end{equation}

Луч пересекается со сферой, если точка луча удовлетворяет соотношению сферы, в таком случае получается система из уравнений \ref{eq:ray_formula} и \ref{eq:sphere_formula}, в которой $\vec{L} = \vec{P}$:

\begin{equation}
	\label{eq:sphere_system}
	\begin{cases}
		\vec{P} = \vec{O} + t\vec{D} \\
		( \vec{P} - \vec{C} , \vec{P} - \vec{C}) = r^2
	\end{cases}.
\end{equation}

Подставив первое уравнение во второе в системе \ref{eq:sphere_system} и преобразовав, получим квадратное уравнение относительно параметра $t$:

\begin{equation}
	\label{eq:quad_sphere}
	t^2 (\vec{D} , \vec{D}) + 2t (\vec{O} - \vec{C} , \vec{D}) + (\vec{O} - \vec{C} , \vec{O} - \vec{C}) - r^2 = 0.
\end{equation}

При решении уравнения \ref{eq:quad_sphere} можно получить от нуля до двух решений, что будет соответствовать ситуациям, когда луч не пересекает сферу, когда луч касается сферы, и когда луч проходит насквозь, пересекая сферу в двух точках. В последнем варианте следует выбирать ближайшую точку пересечения, которая соответствует меньшему значению параметра.


\subsubsection{Пересечение треугольника и луча}

Так как поверхность всех многогранников можно представить в виде набора треугольников, появляется задача определения пересечения треугольника и луча. Имея вершины треугольника в виде точек $V_0$, $V_1$, $V_2$, любую точку в треугольнике можно представить в виде:

\begin{equation}
	T(u, v) = (1 - u - v)\vec{V_0} + u\vec{V_1} + v\vec{V_2},
\end{equation}

где $u$ и $v$ --- барицентрические координаты, которые должны удовлетворять условиям $u \geq 0, v \geq 0, u + v \leq 1$. Тогда вычисление точки пересечения луча эквивалентно вычислению уравнения $\vec{L} = T(u, v)$, которое записывается как:

\begin{equation}
	\vec{O} + t\vec{D} = (1-u-v)\vec{V_0} + u\vec{V_1} + v\vec{V_2}.
\end{equation}

Перегруппировав члены, получается:

\begin{equation}
	\begin{pmatrix}
	-\vec{D}, & \vec{V_1} - \vec{V_0}, & \vec{V_2} - \vec{V_0} 
	\end{pmatrix}
	\begin{pmatrix}
		t \\
		u \\
		v \\
	\end{pmatrix}
	= \vec{O} - \vec{V_0}
\end{equation}

Обозначим $\vec{E_1} = \vec{V_1} - \vec{V_0}, \vec{E_2} = \vec{V_2} - \vec{V_0}, \vec{T} = \vec{O} - \vec{V_0}$, решая уравнение по методу Крамера можно получить:

\begin{equation}
	\begin{pmatrix}
		t \\
		u \\
		v \\
	\end{pmatrix}
	=
	\frac{1}{((\vec{D} \times \vec{E}_2), \vec{E}_1)}
	\begin{pmatrix}
		((\vec{T} \times \vec{E}_1), \vec{E}_2) \\
		((\vec{D} \times \vec{E}_2), \vec{T}) \\
		((\vec{T} \times \vec{E}_1), \vec{D}) \\
	\end{pmatrix}
\end{equation}

Проверяя, что полученные полученные параметры $u$ и $v$ удовлетворяют условиям $u \in [0,1], v \in [0,1], u + v \leq 1$ можно определить, пересекает ли луч треугольник. Благодаря рассчитанному параметру $t$ можно найти точку пересечения, подставив параметр в уравнение луча.

\subsubsection{Пересечение линзы и луча}

Линза представляет собой тело, ограниченное двумя сферическими поверхностями. Однако алгоритм нахождения пересечения сферы и луча не подходит для использования с линзой, так как будет учитывать лишние лучи, которые пересекаются со сферой, но не пересекаются с линзой. Возможно модифицировать алгоритм, добавив проверку, что точка пересечения действительно лежит на поверхности линзы, а не на мнимой сфере, частью которой является поверхность линзы.

\img{70mm}{lens_intersection}{Схема прохождения луча через линзу}

На рисунке \ref{img:lens_intersection} изображено прохождение луча с вектором направления $vec{L}$ и началом в точке $O$. Луч пересекает линзу в точке $P$, вектор $\vec{H}$ направлен центра линзы $C_l$ в $P$, тогда, зная единичный вектор внешней нормали к плоскости линзы $\vec{n}$ можно сказать, что $P$ лежит на поверхности линзы, если скалярное произведение $(\vec{H}, \vec{n}) \geq 0$, то есть угол между $\vec{H}$ и $\vec{n}$ менее или равен 90 градусов. Для нахождения точки $P$ требуется знать центр сферы $C_s$, его можно найти, зная диаметр линзы $d$, $C_l$ и $\vec{n}$ как:

\begin{equation}
	\vec{C_s} = \vec{C_l} -  \vec{n}\sqrt{R^2 - \left(\frac{d}{2}\right)^2}.
\end{equation}

Таким образом, получив точку $P$ с помощью центра сферы и уравнения луча и проверив ее на принадлежность поверхности линзы, можно узнать пересекается ли луч с линзой.

\clearpage

\subsection*{Вывод}

В данном разделе были описаны требования к программному обеспечению, используемые типы и структуры данных и спроектирован алгоритм обратной трассировки лучей.

\clearpage

