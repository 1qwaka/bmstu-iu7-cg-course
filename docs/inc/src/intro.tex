\section*{ВВЕДЕНИЕ}
\addcontentsline{toc}{section}{ВВЕДЕНИЕ}

Компьютерная графика в наши дни является неотъемлемой частью нашей жизни. Она применяется везде, где нужно создание и обработка изображений, огромную роль
играет компьютерная графика в рекламе и индустрии развлечений \cite{boreskov_cg}. В результате появляется задача создания реалистичных изображений, в которых учитываются оптические эффекты преломления и отражения, а также текстура или цвет. Чем больше физических свойств сцены учитывается, тем более реалистичным получается результат.

Существует множетсво алгоритмов компьютерной графики, которые решают эту задачу. Однако построение более качественных изображений требует больше ресурсов (памяти и времени). Это становится более значимой проблемой при расчете динамических сцен в реальном времени.

Целью данной работы является разработка программного обеспечения для визуализации эффектов искажения изображения, получаемых с помощью двояковыпуклой линзы. Должны предоставляться возможности для изменения положения камеры и источника света, изменения спектральных характеристик источника.

Для достижения поставленной цели необходимо решить следующие задачи:

\begin{itemize}
	\item рассмотреть существующие алгоритмы визуализации сцены, учитывающие такие свойства объектов как преломление и отражение света, выбрать достаточный для решения поставленной задачи;
	\item спроектировать выбранный алгоритм;
	\item разработать программное обеспечение и реализовать выбранные алгоритмы;
	\item провести исследование производительности полученного программного обеспечения.
\end{itemize}

\clearpage

