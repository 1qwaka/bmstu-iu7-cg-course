\section{Исследовательская часть}


\subsection{Цель исследования}

Целью исследования является определение скорости работы реализованного алгоритма генерации изображения.


\subsection{Технические характеристики}

Технические характеристики устройства, на котором выполнялись замеры по времени, представлены далее.


\begin{itemize}
	\item Процессор: Intel(R) Core(TM) i5-9300H CPU 2.40GHz \cite{intel}.
	\item Оперативная память: 24 ГБайт.
	\item Операционная система: Windows 10 Pro 22H2 \cite{windows}.
\end{itemize}


При замерах времени ноутбук был включен в сеть электропитания и был нагружен только системными приложениями.


\subsection{Результаты исследования}

Исследование зависимости времени генерации изображения было проведено в двух вариантах:

\begin{itemize}
	\item в зависимости от размера изображения;
	\item в зависимости от количества объектов на сцене.
\end{itemize}

Для обоих вариантов была использована одна из заранее заготовленных и доступных через пользовательский интерфейс сцен --- сцена <<кубы>>, содержащая 25 кубов.

Результаты исследования зависимости времени генерации от размера изображения представлены на рисунке \ref{img:time_sized}. Соотношение сторон изображения --- один к одному, то есть количество пикселей пропорционально квадрату размера.

\img{100mm}{time_sized}{График зависимости времени отрисовки от размера изображения}

Результаты второго варианта исследования представлены на рисунке \ref{img:time_objects}. Для получения зависимости из сцены по одному удалялись объекты и измерялось время отрисовки для полученной сцены. Исследование проводилось с шириной и высотой изображения, равными 400.


\img{100mm}{time_objects}{График зависимости времени отрисовки от количества объектов}

\clearpage

\subsection*{Вывод}

В данном разделе были проведены исследования производительности и приведены результаты работы полученного программного обеспечения.